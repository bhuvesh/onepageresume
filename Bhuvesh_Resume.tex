\documentclass[US paper]{deedy-resume} % Use US Letter paper, change to a4paper for A4 
\usepackage{multicol}
\usepackage{scalerel}
\usepackage{color}
 
\usepackage{comment}
 


\begin{document}
\definecolor{ashgrey}{rgb}{0.7, 0.75, 0.71}
%----------------------------------------------------------------------------------------
%	TITLE SECTION
%----------------------------------------------------------------------------------------
\setlength{\columnseprule}{1pt}
\def\columnseprulecolor{\color{ashgrey}}


\namesection{Bhuvesh}{Kumar}{ % Your name
\urlstyle{same}\href{mailto:bhuvesh@gatech.edu}{bhuvesh@gatech.edu} | % Your website, LinkedIn profile or other web address
\href{http://www.bhuveshkumar.com}{bhuveshkumar.com} | +1 404 259-4524% Your contact information
}
\vspace{-5pt}



\section{Education} 
\vspace{-4pt}
\runsubsection{School of CS, Georgia Tech}
\descript{| Ph.D. in Computer Science - Interested in Machine Learning and Optimization}
\location{Aug 2017 - Present | Atlanta, Georgia} 

\sectionspace

\runsubsection{IIT, Kanpur}
\descript{| B.Tech in Computer Science and Engineering}
\location{Aug 2013- May 2017 | Kanpur, India |}\locationbol{Cumm. Grade Point: 9.7/10} 

\sectionspace % Some whitespace after the section

%------------------------------------------------

\vspace{-6pt}
\section{Internship Experience}
\vspace{-4pt}
{\runsubsection{ Stochastic Kernel PCA}
\descript{| Research Intern under Dr. Raman Arora, Johns Hopkins University}
\location{May 2016 – Aug 2016 | Baltimore, USA}
\begin{tightitemize}
\item Worked on stochastic methods for Kernel PCA by extending Stochastic PCA methods using non linear feature maps.
\item Used Randomized Fourier features and deterministic features using Taylor series to approximate the kernel evaluation.
\end{tightitemize}

\vspace{1mm}

\runsubsection{Nike, NTC Tour Dubai}
\descript{| Nike, Dubai | Software Development Intern}
\location{May 2015 - June 2015 | Dubai, UAE}
\begin{tightitemize}
\item Developed a website for a Nike event, NTC Tour Dubai integrating social tagging in 360 panoramas.
\item Facebook Graph API, email addresses, and the Twitter APIs were used to fetch data, perform tagging, and sharing.
\item Developed and deployed a ticketing system using Amazon AWS for the same event which was attended by over a 1000 guests.
\item Android application and Database servers were set up to check in guests for the event based on the unique QR codes.
\end{tightitemize}

\vspace{1mm}



\runsubsection{Green Screen and Face recognition}
\descript{| Bank Muscat, Dubai | Software Development Intern}

\location{June 2015 - July 2015| Dubai, UAE}
\begin{tightitemize}
\item Developed an app using Microsoft Kinect implementing background subtraction, gesture recognition, face recognition and detection.
\item Designed a multi-screen setup for deploying the app as a marketing and advertisement tool at public places.
\end{tightitemize}



\vspace{1mm}
\vspace{-2.5pt}


%------------------------------------------------
\section{Research Experience}
\vspace{-4pt}
\runsubsection{Optimizing concave erformance measures}
\descript{| Dr. Purushottam Kar, IITK }
\location{Jan 2017 – May 2017 | Kanpur, India}
\begin{tightitemize}
\item Worked on optimizing concave performance measures in adversarial settings and under delayed feedback.
\item Used Online Mirror Descent by formulating a primal-dual problem and obtained sublinear regret bounds.
\end{tightitemize}
\vspace{1mm}


\runsubsection{Non Convex Methods for Surveillance}
\descript{|Dr. Prateek Jain, Microsoft Research and Dr. Purushottam Kar, IITK}

\location{Aug 2016 – December 2016 | Kanpur, India}
\begin{tightitemize}
\item Used alternating minimization technique to solve the non-convex Robust PCA objective for background subtraction.
\item Extended the Robust PCA for still camera videos to videos with camera motion by devising fast methods for homography estimation.
\end{tightitemize}
\vspace{1mm}


\runsubsection{Automatic Video Surveillance}
\descript{| Dr. H. Karnick, IITK}

\location{Jan 2016 – April 2016 | Kanpur, India}
\begin{tightitemize}
\item Developed methods for entity recognition for traffic surveillance using CCTV footage.
\item Implemented Entity recognition using CRFs and RCNN, and face detection and recognition using Viola-Jones and Neural Nets.
\end{tightitemize}
\vspace{1mm}
\runsubsection{Extreme Multiclass-classification}
\descript{| Dr. Prateek Jain, Microsoft Research and Dr. Purushottam Kar, IITK}

\location{Jan 2016 – July 2016 | Kanpur, India}
\begin{tightitemize}
\item Worked on the extreme classification problems where the number of classes, dimensions and (or) training size scales up to millions.
\item Extended a a local embedding based algorithm for extreme multi labelling problems to extreme multiclass settings.
\end{tightitemize}
\vspace{1mm}
\runsubsection{Sparse Recovery and Optimization}
\descript{| Dr. Prateek Jain, Microsoft Research and Dr. Purushottam Kar, IITK}

\location{Jun 2015 – December 2015 | Kanpur, India}
\begin{tightitemize}
\item Worked on the compressive sensing by studying the algorithms in the field: GraDeS, OPMR, Basis Pursuit, CoSAMP, OMP, and Lasso.
\item Performed experimental and theoretical analysis to quantify the performances under varying dimensionality, sparsity, and noise. 
\end{tightitemize}
\vspace{1mm}

\vspace{-5pt}

\section{Select Projects}

\vspace{5pt}
\begin{tightitemize}
\item \textbf{photoCENTER - Image/Video Processing App:} Developed an open-source multi-platform software to edit videos and images including background extraction capabilities using image processing techniques. \vspace{0.5mm}
\item \textbf{Artify:} Designed a web app in Django for deep neural style transfer written in Caffe.
\vspace{0.5mm}
\item \textbf{ColourIT:} Developed a learning algorithm to automatically colour a grayscale image using multiple regressors and deployed it. \vspace{0.5mm}
\item \textbf{Research Group Website} Designed a package to manage a research group's website by implementing self populating project pages, group members, publications, news, and collaborators using MEAN stack.
\end{tightitemize}


\vspace{-3pt}
\sectionspace % Some whitespace after the section


\vspace{-5pt}
\section{Skills}
\vspace{-10pt}
\begin{multicols}{4}
[
]
\location{Languages} \\
C++ \textbullet{} C \textbullet{} Python \textbullet{} Matlab \\
 \textbullet{} Octave \textbullet{} Awk \textbullet{} C\# \textbullet{} Bash 


\columnbreak

\location{Scientific Libraries} \\
Tensorflow  \textbullet{} scikit-learn \textbullet{} Caffe \\\textbullet{} OpenCV \textbullet{} OpenGL \textbullet{} pandas \\

\columnbreak

\location{General Tools} \\
Git \textbullet{} \LaTeX\ \textbullet{} GNUplot \textbullet{} vim \\
\textbullet{} MySQL \textbullet{} MongoDB
\columnbreak


\location{Webdev} \\
Node.js \textbullet{} web.py  \textbullet{} PHP \\
\textbullet{} Javascript \textbullet{} HTML \textbullet{} CSS 


\end{multicols}
\vspace{-10pt}

\vspace{-3pt}
\section{Awards} 
\runsubsection{}
\location{}
\vspace{5pt}
{
\begin{tightitemize}
\item \textbf{Academic Excellence Award}, IIT Kanpur 14',15',16' (Dean's List)
\item Secured \textbf{All India Rank 269} in JEE Advanced, India 2013 among the 200 thousand selected candidates. 
\item Secured \textbf{All India Rank 321} in JEE Mains 2013 among 1.65 million candidates.
\item Awarded \textbf{Chair's fellowship} by The School of CS, Georgia Tech.
\item Awarded the \textbf{KVPY fellowship 2011} and \textbf{NTSE scholarship 2009} by the Govt. of India.
\item Cleared the \textbf{Mathematics}, \textbf{Informatics}, \textbf{Physics}, and \textbf{Astronomy} Olympiads organised by the Govt. of India.
\end{tightitemize}
}
\vspace{-3pt}
\section{Teaching and Leadership}

\vspace{5pt}
\begin{tightitemize}
\item \textbf{Tutor | ESC101, IIT Kanpur (Fall 16, Spring 17):} Delivered weekly tutorials, supervised the TAs, set exams and lab sessions. \vspace{0.5mm}
\item \textbf{Coordinator | Programming Club, IITK:} Organised various programming contests, Hackathons, summer projects, programming workshops and events for the campus community while managing a team of over 15 secretaries.
\vspace{0.5mm}
\end{tightitemize}




\end{document}